\documentclass[8pt]{extarticle}
\usepackage{multicol}
\usepackage[margin=0.3in]{geometry}
\usepackage{paralist}
\usepackage{parskip}
\usepackage{amsmath}
\usepackage{float}
\usepackage{graphicx}
\usepackage{natbib}

\begin{document}
\begin{multicols}{2}

\section{Limits}
Limit of $f(x)$ is $L$ as $x$ approaches $a$:
\begin{equation}
	\lim_{x \to a} f(x) = L
\end{equation}

Right handed limit, i.e. a limit that is approaching from the positive direction and converging on $a$ is denoted by:

\begin{equation}
	\lim_{x \to a^{+}} f(x) = L 
\end{equation}

Similarly a left handed limit, i.e. one that is approaching from the negative direction and converging on $a$ is denoted by:

\begin{equation}
	\lim_{x \to a^{-}} f(x) = L 
\end{equation}

A couple of additional points about handed limits:
\begin{itemize}
	\item They are useful in the case of ``difficult'' functions such as step functions.
	\item If both the left and right limits are equal then a normal limit exists. 
\end{itemize}

\subsection{Properties of Limits}
\begin{enumerate}
	\item Constants can be factored out: 
	
	\begin{equation}
		\lim_{x \to a} [cf(x)] = c \lim_{x \to a} f(x)
	\end{equation}
	
	\item Limits of a sum or difference is just the limit of their parts:

	\begin{equation}
		\lim_{x \to a} [f(x) \pm g(x)] = \lim_{x \to a} f(x) \pm \lim_{x \to a} g(x)
	\end{equation}
	
	\item Similarly, limits applied to the product and quotient of two functions is just the limit of their parts:

	\begin{equation}
		\lim_{x \to a} f(x)g(x) = \lim_{x \to a} f(x) \lim_{x \to a} g(x)
	\end{equation}
	
	\begin{equation}
		\lim_{x \to a} \left[\frac{f(x)}{g(x)}\right] = \frac{\lim_{x \to a}f(x)}{\lim_{x \to a} g(x)}
	\end{equation}
	
	\item Powers can be factored out:

	\begin{equation}
		\lim_{x \to a} [f(x)]^n = [\lim_{x \to a} f(x)]^n
	\end{equation}
	
	\item Limits of constants are just themselves:
	
	\begin{equation}
		\lim_{x \to a} c = c
	\end{equation}
	
	\item Limits of the variable are simply the limit value itself:

 	\begin{equation}
		\lim_{x \to a} x = a
	\end{equation}
	
\subsection{Continuity}

A function is continuous at $x=a$ if:

\begin{equation}
	\lim_{x \to a} f(x) = f(a)
\end{equation}

A function is continuous over an interval $[a,b]$ if it is continuous at each point in the interval.

If $f(x)$ is continuous at $x=a$ then:

\begin{equation}
\begin{split}
	\lim_{x \to a} f(x) \\
	\lim_{x \to a^+} f(x) \\ 
	\lim_{x \to a^-} f(x)
\end{split}
\end{equation}

Two types of discontinuity:
\begin{itemize}
	\item {\bf Jump Discontinuity:} occurs where graphs have a break in them
	\item {\bf Removable Discontinuity:} occurs where there is a hole in the graph
\end{itemize}

\end{enumerate}

\section{Derivatives}
The derivative of $f(x)$ with respect to $x$ is $f'(x)$. This is formally defined as being:

\begin{equation}
	f'(x) = \lim_{x \to 0} \frac{f(x+h) - f(x)}{h}
\end{equation}

Intuitively this is form the function takes as the difference between two points ($x$ and $x+h$ shrinks to zero).

\subsection{Rules for Computing Derivatives}

\begin{itemize}
	\item The derivative of the sum and difference of two functions is simply the derivative of the respective functions:

	\begin{equation}
		(f(x) \pm g(x))' = f'(x) \pm g'(x)
	\end{equation}
	
	\item Constants may be factored out of the derivative:
	
	\begin{equation}
		(cf(x))' = c f'(x)
	\end{equation}
	
	\item The derivative of a constant is always zero:

	\begin{equation}
		f(x) = c \implies f'(x) = 0
	\end{equation}
	
	\item The power rule can be used to compute the derivative of terms raised to a power:

	\begin{equation}
		f(x) = x^n \implies f'(x) = nx^{(n-1)}
	\end{equation}
	
	\item The product rule can be used to compute the derivative of a the product of two functions:
	
	\begin{equation}
		(f(x)g(x))' = f'(x)g(x) + f(x)g'(x)
	\end{equation}
	
	\item Similarly the quotient rule can be used to compute the derivative of the quotient of two functions:
	
	\begin{equation}
		\left(\frac{f(x)}{g(x)}\right)' = \frac{f'(x)g(x) - f(x)g'(x)}{g(x)^2}
	\end{equation}
	
	\item The chain rule can be used to compute the derivative of more complicated functions where the derivative is a composition of two functions:
	
	\begin{equation}
		(f \circ g)'(x) = f'(g(x))g'(x)
	\end{equation}
	
\subsection{Table of Useful Derivatives}

\begin{table}[H]
\centering
\label{my-label}
\begin{tabular}{l | l}
 $\frac{d}{dx} cos(x)$ &  $-sin(x)$ \\
 $\frac{d}{dx} e^x$ &  $e^x$ \\
 $\frac{d}{dx} a^x$ &  $a^xln(a)$ \\
 $\frac{d}{dx} ln(x)$ &  $\frac{1}{x}$ \\
 $\frac{d}{dx} log_a(x)$ &  $\frac{1}{x ln(a)}$ \\
\end{tabular}
\end{table}

\section{Applications of Derivatives}
\subsection{Critical Points}
$x=c$ is a critical point if $f(c)$ exists and the following is true:

\begin{equation}
\begin{split}
	f'(c) = 0 \\
	f'(c) = \text{doesn't exists}
\end{split}
\end{equation}

\subsection{Minimum and Maximum Values}

\begin{itemize}
	\item {\bf Global minimum:} $f(x) \geq f(c)$ for every $x$ in a domain.
	\item {\bf Local minimum:} $f(x) \geq f(c)$ for every $x$ over an interval.
	\item {\bf Global maximum:} $f(x) \leq f(c)$ for every $x$ in a domain.
	\item {\bf Local maximum:} $f(x) \leq f(c)$ for every $x$ over an interval.
\end{itemize}

\subsection{Extreme Value Theorem}
If $f(x)$ is continuous on $[a,b}$ then there exist two numbers such that $a \eq c$, $d \leq b$ such that $f(c)$ is the global maximum and $f(d)$ is the global minimum.


\subsection{Finding Absolute Extrema}
\begin{itemize}
	\item Verify the function is continuous over the interval.
	\item Find all critical points in the interval
	\item Evaluate critical points and end points.
	\item Identify the extrema
\end{itemize}

\end{multicols}
\end{document}
