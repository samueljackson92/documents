%%% LaTeX Template
%%% This template can be used for both articles and reports.
%%%
%%% Copyright: http://www.howtotex.com/
%%% Date: February 2011

%%% Preamble
\documentclass[paper=a4, fontsize=11pt]{scrartcl}	% Article class of KOMA-script with 11pt font and a4 format
\usepackage[margin=0.7in]{geometry}
\usepackage[english]{babel}															% English language/hyphenation
\usepackage[protrusion=true,expansion=true]{microtype}				% Better typography
\usepackage{amsmath,amsfonts,amsthm}										% Math packages
%\usepackage{color,transparent}													% If you use color and/or transparency
\usepackage[hang, small,labelfont=bf,up,textfont=it,up]{caption}	% Custom captions under/above floats
\usepackage{epstopdf}																	% Converts .eps to .pdf
\usepackage{subfig}																		% Subfigures
\usepackage{booktabs}																	% Nicer tables
\usepackage[pdftex]{graphicx}

%%% Advanced verbatim environment
\usepackage{verbatim}
\usepackage{fancyvrb}
\DefineShortVerb{\|}								% delimiter to display inline verbatim text


%%% Custom sectioning (sectsty package)
\usepackage{sectsty}								% Custom sectioning (see below)
\allsectionsfont{%									% Change font of al section commands
	\usefont{OT1}{bch}{b}{n}%					% bch-b-n: CharterBT-Bold font
%	\hspace{15pt}%									% Uncomment for indentation
	}

\sectionfont{%										% Change font of \section command
	\usefont{OT1}{bch}{b}{n}%					% bch-b-n: CharterBT-Bold font
	\sectionrule{0pt}{0pt}{-5pt}{0.8pt}%	% Horizontal rule below section
	}


%%% Custom headers/footers (fancyhdr package)
\usepackage{fancyhdr}
\pagestyle{fancyplain}
\fancyhead{}														% No page header
\fancyfoot[C]{\thepage}										% Pagenumbering at center of footer
\renewcommand{\headrulewidth}{0pt}				% Remove header underlines
\renewcommand{\footrulewidth}{0pt}				% Remove footer underlines
\setlength{\headheight}{13.6pt}

%%% Equation and float numbering
\numberwithin{equation}{section}															% Equationnumbering: section.eq#
\numberwithin{figure}{section}																% Figurenumbering: section.fig#
\numberwithin{table}{section}

\usepackage[parfill]{parskip}
\usepackage{float}
\usepackage{hyperref}
\usepackage[numbers]{natbib}															% Tablenumbering: section.tab#


%%% Title
\title{
	\vspace{-0.5in} 	\usefont{OT1}{bch}{b}{n}
	 Personal Statement \
}

% Authors
\author{
	\usefont{OT1}{bch}{m}{n} Samuel Jackson
%	\\ \usefont{OT1}{bch}{m}{n} University Of Aberystwyth
%	\\   \texttt{slj11@aber.ac.uk}
}
%
%\author{}
\date{}

\begin{document}
%\maketitle
\section*{Personal Statement}
My enthusiasm for computer science comes from two factors: my natural curiosity towards things I don't understand and my passion for problem solving using smart programming. Computers are complex machines and I am continually fascinated by new ways in which they can be applied them to solving problems in an intelligent way. For me the most exciting part of my subject is learning a new or technique and applying it to solve problems. I want to use my degree solve the hard problems, not the easy ones.

I'm drawn to apply to postgraduate positions in the medical domain for a number of factors. Firstly, the area has a clear and meaningful application towards help people which makes it personally satisfying and a worthwhile endeavour. Secondly, I enjoy the idea of working in a areas on the borders of disciplines. During my undergraduate I took an industrial year out at ISIS Neutron Source where I worked in close collaboration with computer scientists and physicists on neutron scattering analysis software. I found that one of the most rewarding parts of the job was working alongside people from a completely different scientific discipline. Thirdly, I believe that the problems posed in medical analysis offer some of the most challenging problems and is an exciting an active research area in which my academic ambitions can be realised.

Currently, I am working through the fourth year of my degree and undertaking a dissertation project looking into the application of dimensionality reduction techniques to the feature space of both real and synthetically generated mammograms. The end goals of the project are to try and understand the similarity the two datasets under projection, particularly in regards to risk classification.

My undergraduate and placement year has given me excellent programming abilities and problem solving skills suitable to a role



\end{document}
