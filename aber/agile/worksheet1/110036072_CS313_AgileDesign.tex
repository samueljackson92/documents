%%% LaTeX Template
%%% This template can be used for both articles and reports.
%%%
%%% Copyright: http://www.howtotex.com/
%%% Date: February 2011

%%% Preamble
\documentclass[paper=a4, fontsize=11pt]{scrartcl}	% Article class of KOMA-script with 11pt font and a4 format
\usepackage[margin=0.7in]{geometry}
\usepackage[english]{babel}															% English language/hyphenation
\usepackage[protrusion=true,expansion=true]{microtype}				% Better typography
\usepackage{amsmath,amsfonts,amsthm}										% Math packages
%\usepackage{color,transparent}													% If you use color and/or transparency
\usepackage[hang, small,labelfont=bf,up,textfont=it,up]{caption}	% Custom captions under/above floats
\usepackage{epstopdf}																	% Converts .eps to .pdf
\usepackage{subfig}																		% Subfigures
\usepackage{booktabs}																	% Nicer tables
\usepackage[pdftex]{graphicx}

%%% Advanced verbatim environment
\usepackage{verbatim}
\usepackage{fancyvrb}
\DefineShortVerb{\|}								% delimiter to display inline verbatim text


%%% Custom sectioning (sectsty package)
\usepackage{sectsty}								% Custom sectioning (see below)
\allsectionsfont{%									% Change font of al section commands
	\usefont{OT1}{bch}{b}{n}%					% bch-b-n: CharterBT-Bold font
%	\hspace{15pt}%									% Uncomment for indentation
	}

\sectionfont{%										% Change font of \section command
	\usefont{OT1}{bch}{b}{n}%					% bch-b-n: CharterBT-Bold font
	\sectionrule{0pt}{0pt}{-5pt}{0.8pt}%	% Horizontal rule below section
	}


%%% Custom headers/footers (fancyhdr package)
\usepackage{fancyhdr}
\pagestyle{fancyplain}
\fancyhead{}														% No page header
\fancyfoot[C]{\thepage}										% Pagenumbering at center of footer
\renewcommand{\headrulewidth}{0pt}				% Remove header underlines
\renewcommand{\footrulewidth}{0pt}				% Remove footer underlines
\setlength{\headheight}{13.6pt}

%%% Equation and float numbering
\numberwithin{equation}{section}															% Equationnumbering: section.eq#
\numberwithin{figure}{section}																% Figurenumbering: section.fig#
\numberwithin{table}{section}

\usepackage[parfill]{parskip}
\usepackage{float}
\usepackage{hyperref}
\usepackage[numbers]{natbib}															% Tablenumbering: section.tab#


%%% Title
\title{
	\vspace{-0.5in} 	\usefont{OT1}{bch}{b}{n}
	 On Design in eXtreme Programming \
}

% Authors
%\author{
%	\usefont{OT1}{bch}{m}{n} Samuel Jackson
%	\\ \usefont{OT1}{bch}{m}{n} University Of Aberystwyth
%	\\   \texttt{slj11@aber.ac.uk}
%}
%
\author{}
\date{}

\begin{document}

\maketitle
Extreme Programming (XP) challenges the traditional ways in which software systems are designed though removing the reliance on the up-front design approach favoured by classical engineering models. In Martin Fowler's essay ``Is Design Dead?'' \cite{fowler2004design} he ultimately argues against the assertion in the title of the article. He argues that design still happens in XP but the way in which we do design has changed. XP removes the need for the formal, rigorous and bureaucratic approach to design present in earlier software development methodologies and moves towards an emphasis on  putting flexibility and simplicity at the forefront of the process.

Among the more interesting points presented  in the article, Fowler asserts that the ability to carry out evolutionary design relies on what he calls ``enabling practices'', namely continuous integration, testing, and refactoring which allow good design to grow organically as the project evolves by diminishing the cost of change. This is echoed by Robert C. Martin in his classic text ``Clean Code'' \cite{martin2008clean} in which he advocates the necessity of testing to give programmers confidence in their ability to change the design of the system through refactoring without the fear of breaking it. Fowler also includes continuous integration as the means to give confidence to developers at the level of the team instead of the individual.

A major theme throughout the piece is the necessity of simplicity and the ``You aren't going to need it'' (YAGNI) principle in the XP paradigm. The main arguments that Fowler attempts to communicate is that XP encourages us to build the design we need instead of the design we want. To clarify that statement, XP done right naturally gravitates towards the simplest appropriate design. It achieves this by not building anything that isn't relevant to the current iteration's user stories. This prevents spending time and effort on building the wrong thing early when future circumstances a subject to change. Additional complexity leads to a design which is harder to modify, increasing the cost of change, while a design that is  simple makes it easier to understand what needs to be changed.

Perhaps the most interesting sections in the article relate to the applicability of traditional design tools to XP. Here Fowler clearly states that he is a moderate in comparison to contemporaries such as Kent Beck and Robert C. Martin. Hardcore XP advocates tend to  reject the traditional use of a formal design language such as UML in favour of techniques such as CRC cards \cite{wells1999crc, wells1999design} whose primary advantages are their intrinsic simplicity (they take minutes to create), flexibility (multiple designs can be rapidly trialled and discussed), and finiteness (both in the sense of physical size and the fact they aren't meant to be permanent).

Fowler offers a common sense compromise by suggesting that UML has a time and a place within the XP methodology. UML offers an abstraction that facilitates the efficient communication of designs between software engineers and can help when sketching out a scaffold for a system. The major point of contention with using UML in XP is when UML becomes less of a napkin sketch and more of a stone tablet. Diagrams should not be set in stone and should exist only as long as they are useful. There use as a form of documentation is redundant almost as soon as it's written in a methodology that organically evolves over time.

A similar argument is given in regards to the use of design patterns. The contents of the Gang Of Four \cite{gamma1994design} is a gold mine in the hands of a good developer, regardless of the methodology used. The principle argument spouted by XP proponents is the blind overuse of patterns in every vaguely applicable situation (often by junior developers). Such an approach violates the axiom of simplicity. Patterns have as much of a place in XP as they do in any other methodology. The key concept to grasp is that patterns should be used when it makes sense to use them and not before (which would violate YAGNI) and that developers should have the courage to remove a pattern during refactoring should it be deemed overly complex. In short, patterns are good, but not at the sacrifice of simplicity. This issue is humorously summarised by developer Jeff Atwood's rant against the ``Head First Design Patterns'' book \cite{atwood2005head}.

Another key point that the article makes is the mindset of the developer to design in XP. Fowler asserts that we must move away for the concept of ``software architects'' who separate themselves from the herd and occupy themselves with the higher realms of abstraction while leaving other engineers to do the dirty work based on commandments handed down from above. In XP a different mindset is required. Expert designers should coach the less experienced members of the team thereby bettering the abilities of the junior members while promoting team cohesion. 

There is also the implication that although developers may have varying skill sets, all developers are equal within XP. Younger members should be encouraged to contribute to design. This aids with the sense of project ownership which leads into Fowler's later point about the ``The Will to Design''.  Having team members who are enthused by a project and take an active part in design nurtures the courage and will to improve the design of the system. More eyes on the code base means issues are caught and resolved sooner. It doesn't need to be everyone's primary responsibility, but it helps if the people who know the code best are active in maintaining its design.

My own opinion on design in XP closely mirrors that which Martin Fowler puts forward in ``Is Design Dead''. I believe that XP's approach to design is sufficient for building software systems. I even agree that effective design can be achieved by sticking to the more hard-line versions of XP advocated by the likes of Kent Beck, where there is no place for UML and that all design issues are resolved by refactoring exclusively. However I am not content to throw my weight behind such a hard-line view because it strikes me as overly dogmatic and (ironically) restrictive.\textsl{•}

Yes, I would agree that an effective design in a large project can be achieved without a single UML diagram or basic architectural  design but my question is why you would want to limit yourself so harshly? If an idea can best be communicated through the use of a quick UML diagram scrawled on a white board then do it. The essence of the argument XP makes is that they become an issue when they are too tightly adhered to. Likewise, I agree with the point made in the section ``Things that are difficult to refactor in''. In some circumstances their may be scope for adding some anticipatory structures in violation of YAGNI, but the circumstances that may justify this are less frequent than most people think and I strongly agree with the statement that it may have well of been harder had a feature been implemented earlier.

Speaking from my own personal experience on industrial placement I can happily testify first hand that many of the techniques outlined in the article are working in practice. Continuous integration and detailed test coverage give the developer the courage to refactor and redesign as necessary and bypass the red tape of planned design. Likewise I've seen full UML designs that have been created only to become inaccurate and redundant within days of implementation. The problems XP attempts to tackle are very real and the solutions it offers have been conclusively proved to work in industry. 

Failure comes from the teams that do not commit to the principles driving XP and evolutionary growth. That is not to say I believe XP should be followed to the absolute logical extreme, but rather that a team needs to take the core ideas it preaches and define a process that works for them. Common sense should be the compass guiding how a process works. Having faith in the power of refactoring in changes, adhering to the YAGNI and DRY principles, and minimising complexity in all its forms is essential for XP to be successful, however a loose architectural design at the start of project or a UML diagram on the back of a napkin isn't the end of the world providing the design can still evolve according to the enabling practices XP preaches.

In summary, I believe that a successfully implemented XP methodology can flatten the cost of change curve and leads flexibility that would otherwise be unachievable in planned design. Design isn't dead in XP, but the nature of design has shifted towards a more organic process. In XP, it is not simply a reactive implementation that grows from the enabling practices, but a flexiable design as well.

\clearpage
\bibliographystyle{unsrtnat}
\bibliography{references}
\end{document}
