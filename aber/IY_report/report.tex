%%% LaTeX Template
%%% This template can be used for both articles and reports.
%%%
%%% Copyright: http://www.howtotex.com/
%%% Date: February 2011

%%% Preamble
\documentclass[paper=a4, fontsize=11pt]{scrartcl}	% Article class of KOMA-script with 11pt font and a4 format
\usepackage[margin=0.7in]{geometry}
\usepackage[english]{babel}															% English language/hyphenation
\usepackage[protrusion=true,expansion=true]{microtype}				% Better typography
\usepackage{amsmath,amsfonts,amsthm}										% Math packages
%\usepackage{color,transparent}													% If you use color and/or transparency
\usepackage[hang, small,labelfont=bf,up,textfont=it,up]{caption}	% Custom captions under/above floats
\usepackage{epstopdf}																	% Converts .eps to .pdf
\usepackage{subfig}																		% Subfigures
\usepackage{booktabs}																	% Nicer tables

%%% Advanced verbatim environment
\usepackage{verbatim}
\usepackage{fancyvrb}
\DefineShortVerb{\|}								% delimiter to display inline verbatim text


%%% Custom sectioning (sectsty package)
\usepackage{sectsty}								% Custom sectioning (see below)
\allsectionsfont{%									% Change font of al section commands
	\usefont{OT1}{bch}{b}{n}%					% bch-b-n: CharterBT-Bold font
%	\hspace{15pt}%									% Uncomment for indentation
	}

\sectionfont{%										% Change font of \section command
	\usefont{OT1}{bch}{b}{n}%					% bch-b-n: CharterBT-Bold font
	\sectionrule{0pt}{0pt}{-5pt}{0.8pt}%	% Horizontal rule below section
	}


%%% Custom headers/footers (fancyhdr package)
\usepackage{fancyhdr}
\pagestyle{fancyplain}
\fancyhead{}														% No page header
\fancyfoot[C]{\thepage}										% Pagenumbering at center of footer
\renewcommand{\headrulewidth}{0pt}				% Remove header underlines
\renewcommand{\footrulewidth}{0pt}				% Remove footer underlines
\setlength{\headheight}{13.6pt}

%%% Equation and float numbering
\numberwithin{equation}{section}															% Equationnumbering: section.eq#
\numberwithin{figure}{section}																% Figurenumbering: section.fig#
\numberwithin{table}{section}

\usepackage[parfill]{parskip}
\usepackage{float}
\usepackage{graphicx}
\usepackage{hyperref}
\usepackage[numbers]{natbib}															% Tablenumbering: section.tab#


%%% Title
\title{
	\vspace{-1in} 	\usefont{OT1}{bch}{b}{n}
	\huge \strut Industrial Year Report \strut \\
	\Large \bfseries \strut Software Developer at STFC, Rutherford Appleton Laboratory \strut
}

%% Authors
\author{
	\usefont{OT1}{bch}{m}{n} Samuel Jackson
	\\ \usefont{OT1}{bch}{m}{n} University Of Aberystwyth
	\\   \texttt{slj11@aber.ac.uk}
}

\date{\today}

\begin{document}
	\maketitle
	\clearpage
	\tableofcontents
	\clearpage
	\hyperdef{}{introduction}{\section{Introduction}\label{introduction}}

This report details the industrial placement undertaken by the author as
part of a year long placement in industry as part of the Software
Engineering MEng course at Aberystwyth university. The position of the
placement was a junior role as part of the Mantid data analysis
framework development team based at the ISIS facility at Rutherford
Appleton Laboratory in Harwell, Oxfordshire, owned by the Science and
Technologies Facilities Council (STFC) for the duration of a one year
contract of employment. This position was then extended by an additional
two months to continue with existing and new project commitments.

ISIS is a world leading neutron and muon scattering facility. The
facility operates a 800 MeV proton synchrotron which acts as a source of
neutrons for both neutron and muon spectroscopy {[}1{]}. Neutron
spectroscopy is used to probe the structure and dynamics of materials at
the fundamental level.

The Mantid project aims to provide a single, unified application for the
analysis of neutron and muon scattering facilities such as ISIS {[}2{]}.
The project is primarily developed by two teams of developers, one based
at ISIS and one at the Spallation Neutron Source (SNS) at Oakridge
laboratory in Tennessee, USA.

\section{Organisational Environment}\label{organisational-environment}

\subsection{Organisation of STFC}\label{organisation-of-stfc}

The Science and Technologies Facilities Council is a UK government
funded body that carries out a wide variety of scientific research
across a multitude of disciplines including particle physics, nuclear
physics, space science and engineering, medical and biological sciences,
and computational science.

While the organisation is funded by the government, it is classified as
a non-governmental body which acts as an umbrella organisation for an
array of facilities based across the UK. These include (but are not
limited to) the central laser facility, diamond light source (which is a
publically limited company of which STFC holds an 86\% share), ISIS
neutron source and RAL space based at Rutherford Appleton Laboratory in
Oxfordshire, the Daresbury Laboratory located in Cheshire and the
Chilbolton Observatory based in Hampshire.

The organisation has its head office located at Polaris House, Swindon,
Wiltshire and is headed by the Chief Executive John Womersley. The
purpose of STFC is to control the general overall management of the
facilities under its control, in particular it is responsible for
allocating budgetary and staffing allowances and liaising between
government departments, particularly the department for business,
innovation and skills.

\subsection{Organisation of ISIS Neutron
Source}\label{organisation-of-isis-neutron-source}

ISIS Neutron source is a project that is owned, operated and funded by
the STFC. The organisational hierarchy of ISIS is headed by the director
Prof.~Robert McGreevy and several division heads for individual
functional areas within ISIS such as the diffraction, spectroscopy and
support, experimental operations, instrumentation, design, and
accelerator divisions.

ISIS is also split into a number of research, operations, and
experimental support groups. The computing group, of which the Mantid
project is a part of, falls into the experimental support group and is
technically a part of the scientific computing group which also includes
the ICAT data catalogue which manages the data collected from sample
runs on the instruments for future analysis.

While this description provides an overview of how the staffing of ISIS
can be divided, in practice there tends to be a lot of cross over
between sections of the organisation depending on a employees skills and
responsibilities. For example, while I was employed as part of the
scientific computing group, my line manager and senior manager were both
members of the molecular spectroscopy research group whose interests I
was responsible for within Mantid.

\subsection{Organisation of the Mantid Project Development
Team}\label{organisation-of-the-mantid-project-development-team}

The Mantid development team in the UK is a subset of the scientific
computing group of ISIS. The Mantid group is headed by a single project
manager (Nick Draper) who is based at ISIS where the project first
started and is responsible for the overall management and direction of
the project. The project is split into two teams, one based at ISIS and
one at the SNS in Oakridge, Tennessee. Both teams consist of a single
lead developer and several senior developers who oversee the major
technical developments and help to guide and manage the rest of the
development team. Both teams in the UK and the US also have there own
manager, but the project manager based at ISIS is in overall control.
Within Mantid, the project manager and the majority of the senior
developers are actually contractors from Tessella Ltd, based in
Abingdon, Oxfordshire. The rest of the development team are directly
employed by ISIS, or in the case of the Americans, by the SNS.

While the author was primarily situated within the development team
based at ISIS, many developers within the team are generally also
attached to a specific scientific group at the facility. In the author's
case this was the ISIS Molecular Spectroscopy group (MSG) {[}3{]} which
specialises in using indirect geometry spectrometers to preform
quasi-elastic, inelastic, and deep inelastic (also know as Compton)
neutron scattering {[}4, 5{]} for condensed matter science and thus much
of the work done over the course of the year was carried out in relation
to the needs and requirements of the MSG.

\section{Technical and Application
Environments}\label{technical-and-application-environments}

The development team based in the UK consisted of a single office
located within the main office building of the ISIS facility. This
office consisted of approximately 13 workstations for use by the team.
The number of machines varied throughout the course of the year
depending on the level of staffing available for the project. Each of
the machines were reasonably powerful 64-bit Dell workstations with
between 8-16 Gb of RAM and 8-16 core intel i7 processors. Typical hard
drive space for the machines was between 512 Gb to 1 Tb with the
majority still being disk drives, but some of the newer machines had
flash based storage.

The operating system that each machine ran was completely left to the
preference of the developer, but it was recommended that the developer
run one of the operating systems supported by Mantid for obvious
reasons. In practice this meant that there was a good variety of
developers using different platforms. The author chose to run Ubuntu
12.13 as his operating system of choice for the majority of development
work, with a dual parititon running Windows 7 which could be swapped to
when circumstances required. Other operating systems used by developers
in the team included Windows 8, Mac OSX Mountain Lion and Mavericks, Red
Hat Enterprise Linux 6, and Fedora 20.

Apart from the workstations, the development also had a collection of
Jenkins build servers in order to support a continuous integration and
testing workflow in conjunction with the Gitflow workflow {[}6{]}. The
build servers and jointly located at both ISIS and the SNS. At the start
of the placement, the build servers for ISIS and the SNS were completely
separate and located at different web address. Each individual build was
run as a single job on the Jenkins build servers. At the beginning of
the current year this was changed so that the servers were located at
the same web address and the organisation of the build servers were
changed to make use of matrix builds. This is where multiple builds are
are kicked off at the same time under a single umbrella job. For example
the development branch matrix build would build the project and run the
unit test on each officially supported OS.

Like the choice of operating system, the development software used by
the team was flexible and open to developer preference. The project is
built using the CMake build system on all supported platforms. On
windows platforms the only supported compiler was Visual Studio 2012 or
2014 and most developers either chose to use the Visual Studio IDE, the
Qt Creator IDE, or the Eclipse IDE. On Mac the Intel C++ compiler is
used and typical IDEs are XCode, Eclipse, or Qt Creator. On Linux
distributions the GNU compiler is the main supported compiler, with
either Eclipse or Qt Creator used as the IDE for development. Many Linux
developers are also content to just use the make command to build the
project from the command line. This approach is often used in
conjunction with lightweight editors such as vim or sublime text.

Unit tests are optionally built along side the project using a separate
build target generated by CMake using the CxxTest unit testing
framework. System tests are written in python and make use of a
collection of custom scripts loosely based on the unittest python module
and makes use of the Mantid applications python API. Debugging software
used typically makes use of Visual Studio on Windows, XCode on Mac and
GDB on Linux distributions.

The Mantid application make use of data files produced directly from
neutron spectrometers. These files are collected on a collection of
servers based by ISIS and owned by the scientific computing department
and provide both the instrument scientists and visiting scientists with
access to the their data. The development team also has direct access to
all of the data generated from the instruments. This are available
through nextwork drives. On Windows operating systems access is provided
though the in-built network drive capabilities. On Mac and Linux access
is obtained through using Samba software in conjunction with the SMB
protocol. Copies of actual instrument data are frequently used as part
of test scripts, especially in the case where the data required for the
test cannot be easily simulated programatically.

\section{Description of Job Role and Work
Done}\label{description-of-job-role-and-work-done}

As mentioned in section \hyperref[introduction]{Introduction}, the
majority of my year was spent attached to the Molecular Spectroscopy
Group (MSG) at ISIS. My role in the development team was to satisfy the
data analysis requirements of the MSG within Mantid. This included
involvement in every part of the development cycle, from gathering
requirements from the users (the instrument scientists in the MSG)
through to implementation of requested features, to testing and
maintenece/bug fixing.

\section{Brief Introduction to Neutron
Scattering}\label{brief-introduction-to-neutron-scattering}

In order to fully understand the work done, a small amount of background
knowledge of the techniques used in neutron scattering experiments are
required along with some definitions of key terms. Broadly speaking,
neutron scattering can be split into two categories: elastic and
inelastic. elastic scattering is where the final energy of a scattered
neutron is equal to the energy of the incident neutron, i.e.~there is no
transfer of energy to or from the sample. Inelastic scattering (which is
what is principally measured by instruments belonging to the MSG) is the
more complex case where the energy of the incident neutron and the
scattered neutron are not equal, i.e.~there is a transfer of energy to
or from the sample. From this transfer in energy and from known
parameters of the instrument an instrument independent scattering
function can be defined usually denoted as $S(Q,\omega)$, where $Q$ is
the momentum transfer and $\omega$ is energy transfer.

\section{Critical Evaluation of
Placement}\label{critical-evaluation-of-placement}

\section*{References}\label{references}
\addcontentsline{toc}{section}{References}

1. Source. IN. 2014. How iSIS works

2. Project M. 2013. Mantid: Manipulation and analysis toolkit for
instrument data

3. Source IN. 2014. Molecular spectroscopy group

4. Sivia DS. 2011. \emph{Elementary scattering theory for x-ray and
neutron users}. OUP Oxford. pp. ed.

5. Fernandez-Alonso F, Price DL. 2013. \emph{Neutron scattering
fundermentals}. Academic Press. pp. 1st ed.

6. Project M. 2014. Git workflow
\end{document}
